% Written by Daina Chiba (daina.chiba@gmail.com).
% It was mostly copied from two poster style files:
% beamerthemeI6pd2.sty written by
%	 	Philippe Dreuw <dreuw@cs.rwth-aachen.de> and
% 		Thomas Deselaers <deselaers@cs.rwth-aachen.de>
% and beamerthemeconfposter.sty written by
%     Nathaniel Johnston (nathaniel@nathanieljohnston.com)
%		http://www.nathanieljohnston.com/2009/08/latex-poster-template/
% ---------------------------------------------------------------------------------------------------%
% Preamble
% ---------------------------------------------------------------------------------------------------%
\documentclass[final]{beamer}
\usepackage[orientation=landscape,size=custom,width=110,height=86,scale=1.2,debug]{beamerposter}
\mode<presentation>{\usetheme{RicePoster}}
\usepackage[english]{babel}
\usepackage[latin1]{inputenc}
\usepackage[T1]{fontenc}
\usepackage{amsmath,amsthm, amssymb, latexsym}

\usepackage{array,booktabs,tabularx}
\newcolumntype{Z}{>{\centering\arraybackslash}X} % centered tabularx columns


%   Begin Additional Packages
\usepackage{blindtext}
\usepackage{graphicx}
\usepackage{tikz}
%   End Additional Packages


% comment
\newcommand{\comment}[1]{}

\newlength{\columnheight}
\setlength{\columnheight}{80cm}
\newlength{\sepwid}
\newlength{\onecolwid}
\newlength{\twocolwid}
\newlength{\threecolwid}
\newlength{\restofpage}
\setlength{\sepwid}{0.024\paperwidth}
\setlength{\onecolwid}{0.24\paperwidth}
\setlength{\twocolwid}{0.4\paperwidth}
\setlength{\threecolwid}{0.19\paperwidth}
\setlength{\restofpage}{0.7\paperwidth}

% ---------------------------------------------------------------------------------------------------%
% Title, author, date, etc.
% ---------------------------------------------------------------------------------------------------%
\title{\huge deton8: Detector of Nuclei}
\author{Will LeVine \& Gabriel Vacaliuc}
\institute[Rice University]{Department of Computer Science, Rice University}
\date[Apr.2018]{April, 2018}
\def\conference{COMP540 Term Project: 2018 Data Science Bowl}
\def\yourEmail{wvl1@rice.edu,gv8@rice.edu}


% ---------------------------------------------------------------------------------------------------%
% Contents
% ---------------------------------------------------------------------------------------------------%
\begin{document}
\begin{frame}[t]

\begin{columns}[t]

\begin{column}{\onecolwid}

    \begin{block}{Introduction}
      \begin{itemize}
        \item Modern medicine generate lots of data that needs to be processed
        \item This often involves hand labeling data such as images
        \item It is desirable to automate this process
        \item One example is labeling microscopic nuclei in images
      \end{itemize}
    \end{block}

    \vskip3ex

    \begin{block}{Preprocessing \& Data Whitening}
        \begin{columns}[c]
            \column{.49\textwidth}
            \begin{itemize}
                \item since the input data is of a range of sizes, we reshape all
                    images to (256, 256)
                \item we negate all images with a white background so we can
                    assosciate brightness with nuclei
                \item since the color channels are highly correlated, we whiten
                    the data, resulting in greyscale images with reduced
                    modality effects
            \end{itemize}

            \column{.02\textwidth}

            \column{.49\textwidth}

            \begin{figure}
                \centering
                \caption{3D Plot of a sample of the training set.}
                \label{fig:correlated-features}
                \includegraphics[width=\textwidth]{../paper/figs/correlated-features.png}
            \end{figure}
        \end{columns}

        \begin{figure}
            \centering
            \caption{Original Data.}
            \label{fig:correlated-features}
            \includegraphics[width=\textwidth]{./figs/dsbowl18-imagegrid-1x3.png}
        \end{figure}

        \begin{figure}
            \centering
            \caption{Inverted and Whitened Data.}
            \label{fig:correlated-features}
            \includegraphics[width=\textwidth]{./figs/dsbowl18-imagegrid-1x3-whitened.png}
        \end{figure}

    \end{block}

    \vskip3ex

    \begin{block}{Hand Designed Features}
      \begin{itemize}
        \item Bilateral Filter: convolves image with weighted Gaussian kernel;
            denoises the image, while still preserving the edges
        \item 50/99 Image Rescaling: stretches pixel distribution; increases
            contrast between foreground and background
        \item Contrast Limited Adaptive Histogram Equalization: increases
            contrast locally; increases brightness of small nuclei
        \item Dilation: convolves image with uniform kernel; increases area of
            small nuclei
      \end{itemize}
    \end{block}

\end{column}

% -----------------------------------------------------------
% Start the second column
% -----------------------------------------------------------
\begin{column}{\restofpage}

    \begin{block}{Pipeline}
        \begin{tikzpicture}
        \node[inner sep=0pt] (raw) at (0,-12)
            {\includegraphics[width=8cm]{./figs/raw.png}};
        \node[inner sep=0pt] (bilateral) at (12,0)
            {\includegraphics[width=8cm]{./figs/bilateral.png}};
        \draw[->,line width=4pt] (raw.east) -- (bilateral.west);
        \node[inner sep=0pt] (rescale) at (12,-9)
            {\includegraphics[width=8cm]{./figs/rescale.png}};
        \draw[->,line width=4pt] (raw.east) -- (rescale.west);
        \node[inner sep=0pt] (clahe) at (12,-18)
            {\includegraphics[width=8cm]{./figs/clahe.png}};
        \draw[->,line width=4pt] (raw.east) -- (clahe.west);
        \node[inner sep=0pt] (dilation) at (12,-27)
            {\includegraphics[width=8cm]{./figs/dilation.png}};
        \draw[->,line width=4pt] (raw.east) -- (dilation.west);
        \node[inner sep=0pt] (sgd) at (24,-9)
            {\includegraphics[width=8cm]{./figs/sgd.png}};
        \draw[->,line width=4pt] (rescale.east) -- (sgd.west);
        \node[inner sep=0pt] (pa) at (24,-18)
            {\includegraphics[width=8cm]{./figs/pa.png}};
        \draw[->,line width=4pt] (clahe.east) -- (pa.west);

		\draw[fill=black,opacity=0.2,draw=black] (2.75,1.25) -- (3.75,1.25) -- (3.75,2.25) -- (2.75,2.25) -- (2.75,1.25);
		\draw[fill=black,opacity=0.2,draw=black] (2.5,1) -- (3.5,1) -- (3.5,2) -- (2.5,2) -- (2.5,1);
		\draw[fill=black,opacity=0.2,draw=black] (2.25,0.75) -- (3.25,0.75) -- (3.25,1.75) -- (2.25,1.75) -- (2.25,0.75);
		\draw[fill=black,opacity=0.2,draw=black] (2,0.5) -- (3,0.5) -- (3,1.5) -- (2,1.5) -- (2,0.5);
		\draw[fill=black,opacity=0.2,draw=black] (1.75,0.25) -- (2.75,0.25) -- (2.75,1.25) -- (1.75,1.25) -- (1.75,0.25);
		\draw[fill=black,opacity=0.2,draw=black] (1.5,0) -- (2.5,0) -- (2.5,1) -- (1.5,1) -- (1.5,0);

        \end{tikzpicture}
    \end{block}

\end{column}

\end{columns}

\end{frame}
\end{document}
